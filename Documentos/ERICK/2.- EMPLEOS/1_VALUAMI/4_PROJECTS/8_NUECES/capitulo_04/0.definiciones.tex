\chapter{DEFINICIONES.} %(((

\section{CONCEPTOS VALUATORIOS.} % (((
En el presente dictamen se utilizan conceptos identificados por siglas, cuyo significado a continuación se explica. \\[6mm]
\textbf{Enfoque de Costos} \\ 
El enfoque de costos parte de estimar el costo del bien valuado a partir de su valor de reposición nuevo o valor de reproducción nuevo, al cual se le calcula y deduce la depreciación correspondiente de acuerdo con el tipo de bien en cuestión, para obtener los diferente rangos de valor a estimar. \\[6mm]
\textbf{Enfoque de Mercado} \\ 
Es la estimación de valor por medio del análisis y comparación en el mercado de ventas recientes de un bien igual o similar al valuado para concluir en el precio más probable de venta de este. \\[6mm]
\textbf{Enfoque de Ingresos} \\ 
Este enfoque de ingresos consiste en estimar un indicativo del valor del bien con base al valor presente de los beneficios futuros derivados del bien y es generalmente medido a través de la capitalización de un nivel especifico de ingresos. 
Se deberá considerar, debidamente fundamentada y soportada, la tasa de capitalización utilizada.
Para la valuación de los bienes, principalmente se deberán considerar la renta o los ingresos que generaría la maquinaria y equipo y no los aplicables al negocio en general, ya que éstos involucran otros bienes e intangibles necesarios para el funcionamiento del negocio o empresa. 
Aunque siempre deberá tomarse en cuenta, este enfoque sólo será aplicable cuando estén claramente identificados en forma separada los ingresos del bien. \\[6mm]
\textbf{Valor de Reposición Nuevo (V.R.N.)} \\ 
Es el costo a precios actuales, de un bien nuevo similar, que tenga la utilidad o función equivalente más próxima al bien que se está valuando, con las características que la  técnica  hubiera  introducido dentro de los modelos considerados equivalentes. \\[6mm]
\textbf{Valor de Neto de Resposición (V.N.R.)} \\ 
Se entiende como el valor que tienen los bienes en la fecha de referencia y se determina a partir del valor de reposición nuevo o del valor de reproducción nuevo, disminuyendo los efectos debidos al deterioro físico y a la obsolescencia funcional y económica de cada bien valuado. \\[6mm] 
\textbf{Valor de Mercado (V.M.)} \\ 
Es la cantidad estimada, en términos monetarios, a partir del análisis y comparación de bienes iguales o similares que han sido vendidos o que se encuentran en proceso de venta en el mercado correspondiente. \\[6mm]
\textbf{Valor Razonable (V.R)} \\ 
Es la cuantía estimada por la que un activo o pasivo debería intercambiarse en la fecha de valuación entre un comprador dispuesto a comprar y un vendedor dispuesto a vender, en una transacción libre, tras una comercialización adecuada en la que las partes hayan actuado con conocimiento, de manera prudente y sin coacción. \\ 
Las especificaciones de esta definición general se pueden encontrar en los numerales 30.1 al 30.7 de la citada Norma Internacional de Valuación generada por el organismo International Valuation Standards Council. \\[6mm] 
\textbf{Valor de Inversión o Valía (V.I.)} \\ 
Es el valor de un activo para el propietario o un propietario potencial, para objetivos individuales de inversión o de operaciones. \\ 
El Valor de inversión o Valía es una base de valor específica de una entidad. 
Aunque el valor de un activo para su propietario puede ser el mismo que la cantidad que puede obtener de su venta a otra parte, esta base de valor refleja los beneficios recibidos por una entidad de poseer el activo y, por tanto, no implica necesariamente un intercambio hipotético. 
El Valor de inversión refleja las circunstancias y objetivos de la entidad para la cual la valuación se está produciendo. 
Es usado con frecuencia para medir el desempeño de una inversión. \\[6mm] 
\textbf{Valor de Liquidación Forazada (V.L.F.)} \\ 
Es el monto que se espera obtener por concepto de una venta pública debidamente anunciada y llevada a cabo en el mercado abierto en la que el vendedor se ve en la obligación de vender a corto plazo ``tal como está y donde está'' el bien. \\[6mm] 
\textbf{Edad Efectiva (E.E.)} \\ 
Es la edad aparente de un activo en comparación con un activo nuevo similar. Frecuentemente calculada mediante la deducción de la vida útil remanente del activo de la vida útil normal. \\[6mm]
\textbf{Vida Útil (V.U.)} \\ 
Es la vida, usualmente en término de años, que un bien durará antes de deteriorarse hasta llegar a una condición en la cual no pueda ser utilizado.
Es derivada de estadísticas de durabilidad y el estudio de bienes específicos bajo condiciones actuales de operación. \\[6mm] 
\textbf{Vida Útil Remanente (V.U.R.)} \\ 
Es la vida física remanente que se estima que tendrá un bien, en condiciones aceptables de utilización.
Se calcula deduciendo la edad efectiva del bien de los de la vida normal. \\[6mm] 
\textbf{Depreciación (D.)} \\ 
Es la pérdida real de un bien debido a su antigüedad, desgaste físico, servicio, uso, obsolecencia funciona, etc. \\ 
Los factores que determinan la depreciación son:
\begin{itemize}
	\item Deterioro físico.
	\item Obsolecnecia funcional.
	\item Obsolecencia económica.
\end{itemize}
Este es el orden en el cual se debe ir afectando el valor de reposición nuevo por cada factor de depreciación estimado. \\[6mm] 
\textbf{Deterioro Físico  (D.F.)} \\ 
Es una forma de depreciación donde la péridda de valor o utilidad es atribuible únicamente a causas físicas como el uso, desgaste y la exposición a los elementos. \\[6mm] 
\textbf{Obsolecencia Funcional.} \\ 
Es una forma de depreciación en la cuál la pérdida de valor es debida a factores inherentes al bien mismo y a cambios en diseño, materiales o procesos resultando principalmente en costos excesivos de operación, así como falta de adecuación, exceso de construcción o materiales, falta de utilidad funciona, etc. \\[6mm] 
\textbf{Obsolecencia Económica.} \\ 
Es una forma de depreciación en la cual la pérdida de valor es causada por condiciones externas desfavorables del entorno económico en el cual se valúa el equipo o maquinaria. \\[6mm] 
\textbf{Depreciación Curable.} \\ 
Depreciación en forma de deterioro u obsolecencia que es económicamente viable de remediar por que el incremento resultante en la utilidad del bien y el valor de éste es igual o mayor que el dato requerido. \\[6mm] 
\textbf{Depreciación Incurable.} \\ 
Depreciación en forma de deterioro u obsolecencia que no es económicamente viable de remediar por que el incremento resultante en la utilidad del bien y el valor de éste es igual o mayor que el gato requerido. \\[6mm] 
\textbf{Principio de Sustitución.} \\ 
El valor de reemplazo de una propiedad es establecido por el valor de otra igual o similar y que presente la misma función, lo cual está fundamentado en que no es razonable que un comprador pague más por una propiedad que está siendo valuada que el costo de un equivalente sustituto. \\[6mm] 
\textbf{Valores Estimados Considerando los Bienes Desinstalados.} \\ 
Se considera al bien desinstalado cuando será trasladado a otra localización para su uso y por tanto las instalaciones y gastos relacionados de transportación y montaje, no contribuyen al valor del bien. \\[6mm] 
\textbf{Valores Estimados Considerando los Bienes para ser Removidos.} \\ 
Se considera al bien para ser removido cuando será trasladado a otra localzación para su uso y por tanto las instalaciones y gastos relacionados de transportación y montaje, no contribuyen al valor del bien.
Así como los gatos y riesgos de la remoción traslado y reinstalación correrán por cuenta del posible comprador. \\[6mm] 
\textbf{Valores Estimados Considerando los Bienes ya Removidos.} \\ 
Se considera al bien ya removido cuando se encuentre desinstalado listo para ser trasladado a otra localización para su uso y por tanto las instalaciones y gastos relacionados de transportación y montaje en que incurrieron originalmente no contribuyen al valor del bien.
Así como los gastos y riesgos del traslado y reinstalación correrán por cuenta del posible comprador.
Se considera un castigo adicional en el valor estimado por tratarse de la compra de un bien al cual no se le pueden hacer pruebas necesarias para verificar su correcto funcionamiento.
% )))

% )))
