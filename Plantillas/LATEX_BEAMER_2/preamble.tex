 
% === DATA === (((
\title{<++>}
\author{Por Erick I. Rodríguez Juárez.}
\institute{UAA. LMA.}
\date{\today}
% \author{\colorbox{titlecolor!20}{Por Erick I. Rodríguez Juárez.}}
% \institute{\colorbox{titlecolor!20}{UAA. LMA.}}
% \date{\colorbox{titlecolor!20}{\today}}
% )))

% === PACKAGES === (((
\usepackage{amsfonts}
\usepackage{amsmath}
\usepackage{amssymb}
% \usepackage{expl3}
% \usepackage{mathrsfs}
\usepackage{graphicx}
% )))

% === FONT === (((
\usefonttheme{professionalfonts}
\usefonttheme{serif}
\usepackage{fontspec}
\setmainfont[
  BoldFont       = bodonibi,
	ItalicFont     = Century modern italic2.ttf,
	BoldItalicFont = bodonibi,
	SmallCapsFont  = lmromancaps10-regular.otf
]{Century_modern.ttf}
% \usepackage{expl3}
\DeclareSymbolFont{italics}{\encodingdefault}{\rmdefault}{m}{it}
\DeclareSymbolFontAlphabet{\mathit}{italics}
\ExplSyntaxOn
\int_step_inline:nnnn { `A } { 1 } { `Z }
 {  \exp_args:Nf \DeclareMathSymbol{\char_generate:nn{#1}{11}}{\mathalpha}{italics}{#1} }
\int_step_inline:nnnn { `a } { 1 } { `z } {  \exp_args:Nf \DeclareMathSymbol{\char_generate:nn{#1}{11}}{\mathalpha}{italics}{#1}}
\ExplSyntaxOff
% )))

% === COMMANDS === (((
\newcommand{\micita}[1]{\([\)\cite{#1}\(]\)}
\newcommand{\dis}{\displaystyle}
\renewcommand{\qed}{\hspace{0.5cm}\rule{0.16cm}{0.4cm}}
\newcommand\Myref[1]{
  \begingroup
  \usebeamerfont*{item projected}%
  \usebeamercolor[bg]{item projected}%
  \begin{pgfpicture}{-1ex}{0ex}{1ex}{2ex}
    \pgfpathcircle{\pgfpoint{0pt}{.75ex}}{1.2ex}
    \pgfusepath{fill}
    \pgftext[base]{\color{fg}\ref{#1}}
  \end{pgfpicture}%
  \endgroup
}
\newcommand{\operator}[1]{\mathop{\vphantom{\sum}\mathchoice
{\vcenter{\hbox{\huge $#1$}}}
{\vcenter{\hbox{\Large $#1$}}}{#1}{#1}}\displaylimits}
\newcommand{\suma}{\operator{\includegraphics[scale=0.09]{IMAGES/TITLEPAGE/Sigma.png}}}
\setlength{\parindent}{0mm}
% )))

% === BEAMER TEMPLATE === (((
% DESPLIEGUE EN VARIAS DIAPOSITIVAS.
% display un item a la vez
% \beamerdefaultoverlayspecification{<+(1)->}
% TITULOS EN SMALL CAPS
\setbeamerfont{title}{family=\scshape\huge}
\setbeamerfont{frametitle}{family=\scshape\LARGE}
\setbeamerfont{section in head/foot}{size = \normalsize, family=\scshape}

% sin footline
\setbeamertemplate{footline}{}

% enumerates sin shade
\setbeamertemplate{enumerate items}[circle] % esto lo estoy usando para los simbolos de enumeración.
\setbeamertemplate{itemize items}[circle] % esto lo estoy usando para los simbolos de itemize.
\setbeamertemplate{section in toc}[circle] % esto lo estoy usando para los simbolos de enumeración.

% quitar espacio
% \addtobeamertemplate{titleframe}{}{\vspace{-5mm}}
% \addtobeamertemplate{block begin}{\vskip - \bigskipamount}{}
% \addtobeamertemplate{block end}{}{\vskip - \smallskipamount}
% \addtobeamertemplate{block example begin}{\vskip - \bigskipamount}{}
% \addtobeamertemplate{block example end}{}{\vskip - \smallskipamount}
% \addtobeamertemplate{block definition begin}{\vskip - \bigskipamount}{}
% \addtobeamertemplate{block definition end}{}{\vskip - \smallskipamount}

% PARA VER SECCIONES HORIZONTALMENTE
\setbeamertemplate{headline}{
\leavevmode
\hbox{
\begin{beamercolorbox}[wd=1.02\paperwidth,ht=4ex,dp=2ex]{palette quaternary}
\insertsectionnavigationhorizontal{\paperwidth}{\hskip 0pt plus1filll}{\hskip 0pt plus1filll}
\end{beamercolorbox}
% 
}
% 
}
% SECCIONES EN PÁGINAS.
\setbeamerfont{section title}{parent=title}
\setbeamertemplate{section page}
{
    \begin{centering}
	    \begin{beamercolorbox}[wd= \linewidth ,sep=12pt,center]{secciones}
    \usebeamerfont{frametitle}\insertsection\par
    \end{beamercolorbox}
    \end{centering}
}
% NUMBERS IN BIBLIOGRAPHY.
\setbeamertemplate{bibliography item}[text]
% )))

% === COLORS === (((
\beamertemplatenavigationsymbolsempty % for remove the nav. symb.
\mode<presentation>

% === list of colors === (((

% === 1 === (((
% rgb(128, 128, 255)
% rgb(255, 77, 204)
% rgb(191, 212, 255)
% )))

% === 2 === (((
%  rgb(230, 128, 128)
%  rgb(92,   121,  126)
%  rgb(136,  129,  101)
% )))

% === 3 === (((
% rgb(140, 129, 255)
% rgb(231, 77, 195)
% rgb(51, 230, 153)
% )))

% === 4 === (((
% rgb(4, 57, 100)
% rgb(76, 128, 128)
% rgb(47, 155, 1)
% )))

% === 5 === (((
% rgb(90, 16, 31)
% rgb(243, 77, 65)
% rgb(230, 77, 102)
% )))

% === 6 === (((
% rgb(140, 129, 255)
% rgb(231, 77, 195)
% rgb(51, 230, 153)
% )))

% === 7 === (((
% rgb(22, 44, 61)
% rgb(104, 131, 142)
% rgb(144, 174, 184)
% )))

% === 8 === (((
% rgb(101, 28, 39)
% rgb(92, 125, 15)
% rgb(112, 29, 47)
% )))

% === 9 === (((
% rgb(186, 0, 77)
% rgb(68, 61, 204)
% rgb(200, 69, 7)
% )))

% === 10 === (((
% rgb(128, 128, 102)
% rgb(255, 102, 102)
% rgb(0, 0, 0)
% )))

% === 11 === (((
% rgb(57, 91, 123)
% rgb(205, 51, 102)
% rgb(187, 162, 1)
% )))

% === 12 === (((
% rgb(241, 106, 77)
% rgb(0, 120, 154)
% rgb(55, 189, 178)
% )))

% === 13 === (((
% rgb(4, 57, 100)
% rgb(51, 102, 77)
% rgb(47, 155, 1)
% )))

% === 14 === (((
% rgb(232, 134, 101)
% rgb(0, 0, 70)
% rgb(187, 162, 142)
% )))

% === 15 === (((
% rgb(99, 76, 58)
% rgb(67, 82, 84)
% rgb(95, 86, 81)
% )))

% === 16 === (((
% rgb(47, 40, 94)
% rgb(230, 77, 102)
% rgb(54, 114, 8)
% )))

% === 17 === (((
% rgb(21, 48, 73)
% rgb(68, 98, 98)
% rgb(59, 98, 11)
% )))

% === 18 === (((
% rgb(166, 134, 93)
% rgb(230, 128, 15)
% rgb(253, 245, 21)
% )))

% === 19 === (((
% rgb(231, 41, 113)
% rgb(39, 24, 103)
% rgb(122, 30, 100)
% )))

% === 20 === (((
% rgb(77, 128, 102)
% rgb(179, 128, 128)
% rgb(102, 179, 179)
% )))

% === 21 === (((
% rgb(174, 0, 1)
% rgb(0, 102, 179)
% rgb(230, 102, 0)
% )))

% === 22 === (((
% rgb(141, 129, 245)
% rgb(204, 147, 203)
% rgb(243, 169, 122)
% )))

% === 23 === (((
% rgb(52, 101, 141)
% rgb(218, 167, 11)
% rgb(133, 125, 10)
% )))

% === 24 === (((
% rgb(105, 106, 171)
% rgb(255, 163, 188)
% rgb(168, 173, 209)
% )))

% === 25 === (((
% rgb(105, 106, 171)
% rgb(153, 49, 80)
% rgb(162, 166, 202)
% )))

% === 26 === (((
% rgb(140, 129, 255)
% rgb(231, 77, 195)
% rgb(51, 230, 153)
% )))

% === 27 === (((
% rgb(235, 90, 73)
% rgb(93, 128, 20)
% rgb(254, 214, 7)
% )))

% === 28 === (((
% rgb(116, 45, 41)
% rgb(179, 77, 77)
% rgb(139, 108, 1)
% )))

% === 29 === (((
% rgb(77, 102, 77)
% rgb(77, 77, 77)
% rgb(144, 206, 123)
% )))

% === 30 === (((
% rgb(255, 77, 102)
% rgb(102, 128, 102)
% rgb(235, 202, 143)
% )))

% === 31 === (((
% rgb(13, 71, 155)
% rgb(102, 102, 102)
% rgb(230, 77, 102)
% )))

% === 32 === (((
% rgb(252, 169, 44)
% rgb(26, 128, 179)
% rgb(190, 95, 123)
% )))

% === 33 === (((
% rgb(12, 100, 246)
% rgb(77, 102, 102)
% rgb(152, 206, 25)
% )))

% === 34 === (((
% rgb(51, 102, 77)
% rgb(4, 57, 100)
% rgb(47, 155, 1)
% )))

% === 35 === (((
% rgb(240, 90, 40)
% rgb(53, 37, 126)
% rgb(157, 40, 53)
% )))

% === 36 === (((
%  rgb(230, 128, 128)
%  rgb(92,   121,  126)
%  rgb(136,  129,  101)
% )))

% === 37 === (((
%  rgb(230, 128, 128)
%  rgb(92,   121,  126)
%  rgb(136,  129,  101)
% )))

% === 38 === (((
% rgb(105, 0, 56)
% rgb(0, 77, 77)
% rgb(128, 26, 9)
% )))

% === 39 === (((
% rgb(241, 106, 77)
% rgb(0, 120, 154)
% rgb(55, 189, 178)
% )))

% === 40 === (((
% rgb(102, 128, 179)
% rgb(77, 128, 128)
% rgb(148, 222, 206)
% )))

% === 41 === (((
% rgb(51, 102, 153)
% rgb(236, 102, 102)
% rgb(137, 137, 137)
% )))

% === 42 === (((
% rgb(240, 87, 90)
% rgb(106, 23, 49)
% rgb(248, 156, 143)
% )))

% === 43 === (((
% rgb(47, 60, 152)
% rgb(0, 139, 206)
% rgb(189, 230, 252)
% )))

% === 44 === (((
% rgb(4, 57, 100)
% rgb(76, 128, 128)
% rgb(47, 155, 1)
% )))
% )))

\definecolor{maincolor}{RGB}{<++>} % main color
\definecolor{titlecolor}{RGB}{<++>} % title color

\setbeamercolor{alerted text}{parent=palette secondary}
\setbeamercolor*{palette primary}{   fg=white     , bg=maincolor}
\setbeamercolor*{palette secondary}{ fg=black     , bg=titlecolor!40}
\setbeamercolor*{palette tertiary}{  fg=white , bg=black}
\setbeamercolor*{palette quaternary}{fg=white     , bg=black!20!maincolor}

\setbeamercolor*{titlelike}{parent=palette secondary}
\setbeamercolor{frametitle}{parent=palette primary}

\setbeamercolor*{separation line}{}
\setbeamercolor*{fine separation line}{}
\setbeamercolor{block body}{parent=normal text,use=block title,bg=maincolor!15!white,fg=black}
\setbeamercolor{block title}{bg=maincolor!90!blue,fg=white}

\setbeamercolor{block title example}{bg=maincolor!40!blue,fg=white}
\setbeamercolor{block body example}{parent=normal text,use=block title,bg=maincolor!15!white,fg=black}
\setbeamercolor{item projected}{bg=maincolor}

\setbeamercolor{section title}{parent=titlelike}
\setbeamercolor{secciones}{fg=black,bg=titlecolor!20}
\setbeamercolor{section in head}{parent=palette quaternary}
\setbeamertemplate{section in head/foot shaded}{\color{black!70!maincolor}\usebeamertemplate{section in head/foot}}
\mode<all>
% )))

% === TITLEPAGE === (((
\newcommand{\portada}{
\addfontfeature{LetterSpace=-5}
{
\usebackgroundtemplate{\includegraphics[width=  \paperwidth]{IMAGES/TITLEPAGE/<++>}}
\begin{frame}[plain]
\titlepage
\end{frame}
}
}
% )))
