 
% === DATA === (((
\title{\sc Comandos Básicos de TikZ.}
\author{Por Erick I. Rodríguez Juárez.}
\institute{UAA. LMA.}
\date{\today}
% \author{\colorbox{titlecolor!20}{Por Erick I. Rodríguez Juárez.}}
% \institute{\colorbox{titlecolor!20}{UAA. LMA.}}
% \date{\colorbox{titlecolor!20}{\today}}
% )))

% === FONT === (((
\setbeamerfont{normal text}{series= \ttfamily}
\AtBeginDocument{\usebeamerfont{normal text}}
% )))

% === COMMANDS === (((
\hfuzz=5pt
\newcommand{\tik}{Ti\textit{k}Z }
% )))

% === BEAMER TEMPLATE === (((
% DESPLIEGUE EN VARIAS DIAPOSITIVAS.
% display un item a la vez
% \beamerdefaultoverlayspecification{<+(1)->}
% TITULOS EN SMALL CAPS
\setbeamerfont{title}{family=\scshape\huge}
\setbeamerfont{frametitle}{family=\scshape\LARGE}
\setbeamerfont{section in head/foot}{size = \normalsize, family=\scshape}

% sin footline
\setbeamertemplate{footline}{}

% \setbeamerfont{block title}{series=\sc}
% \setbeamerfont{block body}{series=\ttfamily }
% enumerates sin shade
\setbeamertemplate{enumerate items}[circle] % esto lo estoy usando para los simbolos de enumeración.
\setbeamertemplate{itemize items}[circle] % esto lo estoy usando para los simbolos de itemize.
\setbeamertemplate{section in toc}[circle] % esto lo estoy usando para los simbolos de enumeración.

% quitar espacio
% \addtobeamertemplate{titleframe}{}{\vspace{-5mm}}
% \addtobeamertemplate{block begin}{\vskip - \bigskipamount}{}
% \addtobeamertemplate{block end}{}{\vskip - \smallskipamount}
% \addtobeamertemplate{block example begin}{\vskip - \bigskipamount}{}
% \addtobeamertemplate{block example end}{}{\vskip - \smallskipamount}
% \addtobeamertemplate{block definition begin}{\vskip - \bigskipamount}{}
% \addtobeamertemplate{block definition end}{}{\vskip - \smallskipamount}

% PARA VER SECCIONES HORIZONTALMENTE
\setbeamertemplate{headline}{
\leavevmode
\hbox{
\begin{beamercolorbox}[wd=1.02\paperwidth,ht=4ex,dp=2ex]{palette quaternary}
\insertsectionnavigationhorizontal{\paperwidth}{\hskip 0pt plus1filll}{\hskip 0pt plus1filll}
\end{beamercolorbox}
% 
}
% 
}
% SECCIONES EN PÁGINAS.
\setbeamerfont{section title}{parent=title}
\setbeamertemplate{section page}
{
    \begin{centering}
	    \begin{beamercolorbox}[wd= \linewidth ,sep=12pt,center]{secciones}
    \usebeamerfont{frametitle}\insertsection\par
    \end{beamercolorbox}
    \end{centering}
	\begin{minipage}[t]{0.5\linewidth}
		\tableofcontents[currentsection,sections={1-4}]
		% \tableofcontents[part=0]
	\end{minipage}
	\begin{minipage}[t]{0.45\linewidth}
		\tableofcontents[currentsection,sections={5-6}]
		% \tableofcontents[part=1]
	\end{minipage}
}
\AtBeginSection{\frame{\sectionpage}}
% NUMBERS IN BIBLIOGRAPHY.
\setbeamertemplate{bibliography item}[text]
% )))

% === COLORS === (((
\beamertemplatenavigationsymbolsempty % for remove the nav. symb.
\mode<presentation>

\colorlet{maincolor}{purple}
\colorlet{titlecolor}{blue}

\setbeamercolor{alerted text}{parent=palette secondary, bg=green!50}
\setbeamercolor*{palette primary}{   fg=white     , bg=maincolor}
\setbeamercolor*{palette secondary}{ fg=black     , bg=titlecolor!40}
\setbeamercolor*{palette tertiary}{  fg=white , bg=}
\setbeamercolor*{palette quaternary}{fg=white     , bg=black!20!maincolor}

\setbeamercolor*{titlelike}{parent=palette tertiary}
\setbeamercolor{frametitle}{parent=palette primary}

\setbeamercolor*{separation line}{}
\setbeamercolor*{fine separation line}{}
\setbeamercolor{block body}{parent=normal text,use=block title,bg=titlecolor!15!white,fg=black}
\setbeamercolor{block title}{bg=titlecolor!90!blue,fg=white}

\setbeamercolor{block title example}{bg=maincolor!80!black,fg=white}
\setbeamercolor{block body example}{bg=maincolor!15!white,fg=black}
\setbeamercolor{item projected}{bg=maincolor}

\setbeamercolor{section title}{parent=titlelike}
\setbeamercolor{secciones}{fg=black,bg=titlecolor!20}
\setbeamercolor{section in head}{parent=palette quaternary}
\setbeamertemplate{section in head/foot shaded}{\color{black!70!maincolor}\usebeamertemplate{section in head/foot}}
\mode<all>
% )))

% === TITLEPAGE === (((
\setbeamercolor{author}{fg = white}
\setbeamercolor{institute}{fg = white}
\setbeamercolor{date}{fg = white}
\newcommand{\portada}{
{
\usebackgroundtemplate{\includegraphics[width=  \paperwidth]{IMAGES/TITLEPAGE/blue.pdf}}
\begin{frame}[plain]
\titlepage
\end{frame}
}
}
% )))
