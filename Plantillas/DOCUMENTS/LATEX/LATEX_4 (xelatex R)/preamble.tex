
% === DATA === (((
\author{\textit{Por Erick I. Rodríguez Juárez.}}
% \title{\textbf{ . Álgebra Abstracta.}}
% \title{\textbf{ . Variable Compleja II.}}
% \title{\textbf{ . Topología General.}}
% \title{\textbf{ . Modelos de Enseñanza.}}
% \title{\textbf{ . Control Estadístico de Calidad.}}
% \date{}
% )))

% === PACKAGES === (((
\usepackage{amsfonts}
\usepackage{amsmath}
\usepackage{amssymb}
% \usepackage{expl3}
\usepackage{fullpage}
\usepackage{mathrsfs}
\usepackage{graphicx}
\usepackage{listings} 
% )))

% === FONT === (((
\setmainfont[
  BoldFont       = bodonibi,
	ItalicFont     = Century modern italic2.ttf,
	BoldItalicFont = bodonibi,
	SmallCapsFont  = lmromancaps10-regular.otf
]{Century_modern.ttf}
% )))

% === COMMANDS === (((
\newcommand{\titulo}{
\addfontfeature{LetterSpace=-5}
\pagestyle{empty}
\maketitle 
\thispagestyle{empty}
} 
\newcommand{\dis}{\displaystyle}
\newcommand{\qed}{\hspace{0.5cm}\rule{0.16cm}{0.4cm}}
\newcommand{\operator}[1]{\mathop{\vphantom{\sum}\mathchoice
{\vcenter{\hbox{\huge $#1$}}}
{\vcenter{\hbox{\Large $#1$}}}{#1}{#1}}\displaylimits}
\newcommand{\suma}{\operator{\includegraphics[scale=0.09]{IMAGES/TITLEPAGE/Sigma.png}}}
\setlength{\parindent}{0mm}
% )))

% === FONT IN MATH MODE === (((
\DeclareSymbolFont{italics}{\encodingdefault}{\rmdefault}{m}{it}
\DeclareSymbolFontAlphabet{\mathit}{italics}
\ExplSyntaxOn
\int_step_inline:nnnn { `A } { 1 } { `Z }
 {  \exp_args:Nf \DeclareMathSymbol{\char_generate:nn{#1}{11}}{\mathalpha}{italics}{#1} }
\int_step_inline:nnnn { `a } { 1 } { `z } {  \exp_args:Nf \DeclareMathSymbol{\char_generate:nn{#1}{11}}{\mathalpha}{italics}{#1}}
\ExplSyntaxOff
% )))

% === LST-LISTINGS === (((
\usepackage{xcolor}
\definecolor{backcolour}{rgb}{0.95,0.95,0.92}

\lstdefinestyle{mystyle}{
  backgroundcolor  =  \color{backcolour},
  commentstyle     =  \color{gray},
  numberstyle      =  \tiny\color{gray},
  stringstyle      =  \color{purple},
  basicstyle       =  \ttfamily\footnotesize,
  numbers          =  left,
  numbersep        =  5pt,
  keywordstyle     =  \color{blue},
  identifierstyle  =  \color{orange},
}

\lstset{style=mystyle}
% )))
